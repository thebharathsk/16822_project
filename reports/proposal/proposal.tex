\documentclass[10pt,letterpaper]{article}
\usepackage{setspace}
\usepackage[utf8]{inputenc}
\usepackage[top=1in,bottom=1in,right=1.0in,left=1.0in]{geometry}
\usepackage{times}
\usepackage{graphicx}
\usepackage{amsmath,amssymb} % define this before the line numbering.
\usepackage{color}
\usepackage[breaklinks=true,bookmarks=false,colorlinks=true,citecolor=green,urlcolor=blue,linkcolor=black]{hyperref}
\usepackage{booktabs}
\usepackage{epstopdf}
\usepackage{pdfpages}
\usepackage{hyphenat}
\usepackage{fancyhdr}
\usepackage{wrapfig}
\usepackage[font=small,labelfont=bf]{caption}
\usepackage{booktabs}
\usepackage{multirow}
\usepackage{parskip}
\pagenumbering{gobble}

\title{
\Large{\textbf{Novel View Synthesis for Portrait Images}}
}

\date{}
\author{
  \begin{tabular}{cc}
    Aniruddha Mahapatra & Anujraaj Goyal \\
    \texttt{amahapat@andrew.cmu.edu} & \texttt{anujraag@andrew.cmu.edu} \\
  \end{tabular} \\[0.5em]
  \multicolumn{1}{c}{Bharath Somayajula} \\
  \multicolumn{1}{c}{\texttt{bsomayaj@andrew.cmu.edu}}
}
\usepackage{lastpage}

\begin{document}

\maketitle
\thispagestyle{empty}
\section{Goal}
The problem of novel view synthesis has been studied under various settings. The most common setting consists of generation of novel views of a scene given multiple images of a scene captured from different poses. In this project we aim to synthesize novel views from a single input image in a simplified setting of a portrait image of a person standing in front a far away background. Generation of such novel views can enable editing of portraits to change the viewpoint, generation of live portrait images and better user experience in virtual reality video conferencing applications.

\section{Proposal}
The input image consists of a person standing in front of a far away background. First, the background and foreground are identified using pre-existing human segmentation network. Since the background is far-away we can neglect the depth differences with-in objects at infinity and treat the background as a planar object. This means the background in the novel view and the background in the input view are related by a simple homography. However, this doesn't hold true for the person in the foreground who is not a planar object. To generate the novel foreground, we use a pre-trained human depth estimation network to estimate the depth of the person in the input image. We use this information to project the person into 3D and then render the novel view using a different camera pose. The final novel view is generated by compositing the novel foreground and the novel background.  

That said, any proposal you submit should try to answer a few basic questions:
\begin{itemize}
    \item What is the goal of the project, and why is it worth doing?
    \item What will you need to do to achieve this goal? 
    \item How does your project relate to research/systems that exist?
    \item If you are doing something new in terms of an approach, why are current methods not sufficient? And what are your key ideas/insights that make you believe you might succeed?
    \item How will success be measured -- qualitatively/quantitatively? Are there any datasets/metrics you plan to use for training/evaluation?
\end{itemize}

\end{document}