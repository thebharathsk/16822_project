\documentclass[10pt,letterpaper]{article}
\usepackage{setspace}
\usepackage[utf8]{inputenc}
\usepackage[top=1in,bottom=1in,right=1.0in,left=1.0in]{geometry}
\usepackage{times}
\usepackage{graphicx}
\usepackage{amsmath,amssymb} % define this before the line numbering.
\usepackage{color}
\usepackage[breaklinks=true,bookmarks=false,colorlinks=true,citecolor=green,urlcolor=blue,linkcolor=black]{hyperref}
\usepackage{booktabs}
\usepackage{epstopdf}
\usepackage{pdfpages}
\usepackage{hyphenat}
\usepackage{fancyhdr}
\usepackage{wrapfig}
\usepackage[font=small,labelfont=bf]{caption}
\usepackage{booktabs}
\usepackage{multirow}
\usepackage{parskip}
\pagenumbering{gobble}

\title{
\Large{\textbf{Project Proposal: Amazing Title}}
}

\date{}
\author{
  \begin{tabular}{cc}
    Anirudha Mahapatra & Anujraj Goyal \\
    \texttt{apillay@andrew.cmu.edu} & \texttt{bsomayaj@andrew.cmu.edu} \\
    Bharath Somayajula & Jay Karhade \\
    \texttt{apillay@andrew.cmu.edu} & \texttt{bsomayaj@andrew.cmu.edu}
  \end{tabular} \\[0.5em]
}
\usepackage{lastpage}

\begin{document}

\maketitle
\thispagestyle{empty}

This is blank template, and only serves to constrain the page/margin/font size to ensure readability. Given that the projects/proposals can vary  in scope and goals, there is no constraint on the sections you need to have.

That said, any proposal you submit should try to answer a few basic questions:
\begin{itemize}
    \item What is the goal of the project, and why is it worth doing?
    \item What will you need to do to achieve this goal? 
    \item How does your project relate to research/systems that exist?
    \item If you are doing something new in terms of an approach, why are current methods not sufficient? And what are your key ideas/insights that make you believe you might succeed?
    \item How will success be measured -- qualitatively/quantitatively? Are there any datasets/metrics you plan to use for training/evaluation?
\end{itemize}

\end{document}